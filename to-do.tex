\documentclass[12pt,a4paper,twoside]{article}
\usepackage[T1]{fontenc}      %-->  formato de fontes
\usepackage{ae}               %-->  Corrige problema nas fontes
%\usepackage[brazil]{babel}
\usepackage[utf8]{inputenc}
\usepackage{indentfirst}
\usepackage{color}
\usepackage{pgf}
\usepackage{tikz}
\usepackage{enumerate}
%\usepackage{setspace}
\usepackage{geometry}
\usepackage{amstext}
\usepackage{graphicx}
\usepackage{amsthm}
\usepackage{amsfonts}
\usepackage{amssymb}
\usepackage{amsmath}
\usepackage{esint}
\usepackage{multicol}
\usepackage{multirow}
\usepackage{float}
\usepackage{color}
\usepackage{amscd}
\usepackage{tabularx}
\usepackage{cancel}
\usepackage{cases}
\usepackage[normalem]{ulem} 
\usepackage{soul}
\usepackage[mathscr]{eucal}
\usepackage{hyperref}
\usepackage{bm} %bold for vectors and symbols
%\usepackage{physics}
\usepackage{subcaption}
\usepackage{enumitem}   
\usepackage{mathtools}
\usepackage[]{algorithm2e}
\usepackage{listings}
\usepackage{xcolor}

\definecolor{dkgreen}{rgb}{0,0.6,0}
\definecolor{gray}{rgb}{0.5,0.5,0.5}
\definecolor{mauve}{rgb}{0.58,0,0.82}

\lstset{frame=tb,
  language=C++,
  aboveskip=3mm,
  belowskip=3mm,
  showstringspaces=false,
  columns=flexible,
  basicstyle={\small\ttfamily},
  numbers=none,
  numberstyle=\tiny\color{gray},
  keywordstyle=\color{blue},
  commentstyle=\color{dkgreen},
  stringstyle=\color{mauve},
  breaklines=true,
  breakatwhitespace=true,
  tabsize=3
}
%\usepackage{biblatex} %to use citeonline


%References
%\addbibresource{References.bib}  
\usepackage{natbib} %to use citep


%highlight
\usepackage{alltt}
\definecolor{string}{rgb}{0.7,0.0,0.0}
\definecolor{comment}{rgb}{0.13,0.54,0.13}
\definecolor{keyword}{rgb}{0.0,0.0,1.0}
\usepackage{mdframed}

\usepackage{array}
\newcolumntype{C}[1]{>{\centering\let\newline\\\arraybackslash\hspace{0pt}}m{#1}}
\newcolumntype{L}[1]{>{\let\newline\\\arraybackslash\hspace{0pt}}m{#1}}

\newcommand{\rs}{R\$ \ }
\DeclareMathOperator*{\sen}{sen}
\DeclareMathOperator*{\spn}{span}
\DeclareMathOperator*{\sign}{sign}
\DeclareMathOperator*{\argmin}{argmin}
\DeclareMathOperator*{\posto}{posto}
\DeclareMathOperator*{\diag}{diag}
\newcommand{\E}{\quad \mbox{e}\quad}
\newcommand{\e}{ \; \mbox{e} \; }
\newcommand{\red}[1]{\textcolor{red}{#1}}
\newcommand{\blue}[1]{\textcolor{blue}{#1}}
\newcommand{\green}[1]{\textcolor{green}{#1}}
\newcommand{\gray}[1]{\textcolor{gray}{#1}}
\newcommand{\white}[1]{\textcolor{white}{#1}}
\newcommand{\R}{\mathbb{R}}
\newcommand{\C}{\mathbb{C}}
\newcommand{\res}{\noindent \textit{\underline{Resolução:} }}
\newcommand{\bs}[1]{\boldsymbol{#1}} %\boldsymbol
\newcommand{\bsx}{\textbf{x}}
\newcommand{\bsy}{\textbf{y}}

\newtheorem{theorem}{Teorema}
\newtheorem{definition}[theorem]{Definição}
\newtheorem{proposition}[theorem]{Proposição}
\newtheorem{teor}[theorem]{Teorema}
\newtheorem{teo}{Teorema}[section]
\newtheorem{corollary}[theorem]{Corolário}
\newtheorem{remark}[theorem]{Observação}


%Evaluated symbol
% \DeclarePairedDelimiter\evaluat{.}{\rvert}
% \reDeclarePairedDelimiterInnerWrapper\evaluat{nostar}{%
%     \mathopen{}#2\mathclose{#3}%
% }
% \reDeclarePairedDelimiterInnerWrapper\evaluat{star}{%
%     \mathopen{}\mathclose\bgroup #1\hskip -\nulldelimiterspace \relax
%     #2\aftergroup\egroup #3%
% }


\geometry{
%paperwidth=210mm,
%paperheight=297mm,
%textwidth=150mm,
%textheight=200mm,
top=25mm,
bottom=20mm,
left=25mm,
right=20mm}
\renewcommand{\baselinestretch}{1.1}

\raggedbottom %evita aquele espaço extra

\begin{document}

\thispagestyle{empty}
\begin{center}
\includegraphics[scale=0.4]{UNC_logo-main.jpg}
\end{center}
\begin{center}
University of North Carolina at Chapel Hill\\
% Learning IBAMR \\
\vspace{5cm}
\begin{Large}
\textbf{Learning IBAMR}\\
\end{Large}
\vspace{4.5cm}
\begin{tabular}{llC{4cm}llll}

\end{tabular}\\
\vspace{4.5cm}
Chapel Hill, August 2022.
\end{center}

\newpage
\thispagestyle{empty}
.

% \newpage
% \thispagestyle{empty}
% \section*{Apresentação}
% Nesse relatório apresentamos o problema de Navier-Stokes modelado por um programa desenvolvido em código Fortran 90 e alguns testes que visam avaliar os resultados para que, então, o mesmo seja usado para modelar uma câmara do coração levando em conta a função de espessura $h=h(x,y)$.

% \vskip 0.2 cm
% \par 
 \newpage
\thispagestyle{empty}
%\listoffigures
\subsection*{Current To Do List}
\begin{enumerate}
    \item Iterative Methods
    \item Multigrid
    \item DG
    \item H-Div (H-curl?)
\end{enumerate}

\subsection*{Questions we want to be able to answer:}
\subsubsection*{iterative methods}
- high frequency? what does it mean to have high frequency

- do we ever use preconditioners like PA=Pb
\subsection*{multigrid}
- how do the eigendirections change when we coarsen the grid?
- what guarentees that squashing errors on coarse grid keeps them small on fine grid 

\subsubsection*{H-Div}
- Why do H-Div conforming elements preserve imcompresibility 


\subsection*{Eventually want to learn}

\begin{enumerate}
\item MPI exercises %https://www.mcs.anl.gov/~itf/dbpp/text/node104.html#SECTION035100000000000000000
\item parallel computing
\item type up adams bashforth, stability/consistency, and projection methods
\item saddle point
\item eigenvalues
\item ppm vs spacial centered method (related to slope limiters)
\item spectral methods
\end{enumerate}

\begin{itemize}
    \item Laryssa teaches Bryn cmake
    \item mass lumping - citations in paper by david (shared in slack convo bryn and laryssa)
    \item checkerboard
    \item locking
    \item study data\_idx;
    \item Arena memory;
    \item Find out what object\_name is used for in IBExplicitHierarchyIntegrator;
    \item check that IBFEMethod is the child of IBStrategy
    \item Go over IBTK example, LaplaceSC  and explore: register, index, idx is just an int - how does the level get to know all the information from the int? AllocatePatchData?  etc
    \item [Laryssa] check the classname::classname(input1): d\_input1(input1) and how this fills out/initializes the class attributes
    \item In ibtk/libmesh\_utilities.h check the linear interpolation - line 1034 - intersect line wiht edge
    \item where are the boundary conditions implemented and how to specify boundary conditions outside of the input file 
    \item where are the ghost values for the scratch context filled;
    \item Stability and consistency;
\end{itemize}
\end{document}